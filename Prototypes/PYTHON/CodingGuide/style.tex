\documentclass[10pt,a4paper]{scrartcl}
\usepackage[utf8]{inputenc}
\usepackage{amsmath}
\usepackage{amsfonts}
\usepackage{amssymb}
\usepackage{hyperref}
\title{OOP PYTHON coding style}
\author{Benjamin Rüth}
\subtitle{for the BGCE Project 2015}
\begin{document}
\maketitle
In this we are going to collect coding style guidelines for our project

\section{General Style}
\begin{itemize}
\item follow PyCharm style (PEP8)
\item use meaningful names in any case! No abbreviations without a comment
\end{itemize}

\section{Functions and Variables}
\begin{itemize}
\item Variablenames (CamelCase!): \verb+myVariable+
\item variable and function name start with small letter: \texttt{void myFunc(), myDataType myVariable}
\item class name starts with bigletter: \texttt{class MyClass}
\end{itemize}

\section{Object oriented programming}
\begin{itemize}
\item there is no way for defining private variables in python! the following convention is applied:
\begin{itemize}
\item variables starting with \verb+__+ (e.g. \verb+__myMember+) should not be touched from outside
\item for variable modification and access add appropriate getters and setters
\item from my personal experience and fails: \verb+_+ seems to me to mean something like protected and \verb+__+ derived. Also to the interpreter! This means, if you want to access variables from a base class via a derived class you have to be careful. See the following example:
\begin{verbatim}
class base(object):    
    def __init__(self):
        self._foo = 1

class derived(base):    
    def __init__(self):
        base.__init__(self)
        self._bar = 2
    def print_foo(self):
        print self._foo

a = derived()
a.print_foo()
\end{verbatim}
This doesn't work, if \verb+_foo+ has two leading \verb+_+ instead of one (\verb+__foo+)!
\item see \href{http://stackoverflow.com/questions/1641219/does-python-have-private-variables-in-classes}{this post} for additional information
\end{itemize}

\end{itemize}

\end{document}