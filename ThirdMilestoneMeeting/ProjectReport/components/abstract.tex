% Abstract for the TUM report document
% Included by MAIN.TEX


\clearemptydoublepage
\phantomsection
\addcontentsline{toc}{chapter}{Abstract}	





\vspace*{2cm}
\begin{center}
{\Large \bf Abstract}
\end{center}
\vspace{1cm}

Topology optimization is becoming an increasingly important tool in CAD. Several open-source topology optimization tools already exist, but are generally unsuitable for efficient incorporation in a design process, as there is no straightforward way to reacquire an editable CAD format. For this purpose, the software CADO (Computer Aided Design Optimizer), was developed. The software incorporates a topology optimiser, which works on voxelized CAD designs, and gives back outputs in voxel grid representations. An algorithm was developed to retrieve a CAD-ready surface representation of this data. From the voxel data, a surface is extracted using Dual Contouring. This is reconstructed into a network of tensor product NURBS surface patches using a linear least-squares fitting scheme. The constraint to get smooth ($C^1$) connections between the patches is applied fitting to another network of points, related to the surface by a slightly modified version of the scheme of Peters \cite{peters1992constructing}. The NURBS surface patches are then readily converted to a standard CAD format, and other constraints are taken into account. In this report, we present the theory behind the methods used and the implementations thereof. We conclude with a summary of the capabilities of CADO, and how it can be extended in the future.
%
%
% The main challenge lies in converting mesh-based topology optimized solutions to a smooth geometry representation. We address this issue with a two-stage Dual Contouring surface reconstruction scheme for coarse parametrization patches and fine vertices. A B-Spline surface is fitted by a least-square approach using control points described by the Peters' scheme. This ensures geometrically continuous surfaces. In conclusion, a fully CAD-integrated topology optimization tool-chain is provided.
