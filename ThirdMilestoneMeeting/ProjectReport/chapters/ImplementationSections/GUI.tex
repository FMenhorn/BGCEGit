\section{Graphical User Interface}
\label{sec:gui}
In order to make user interaction with CADO as simple as possible, Graphical User Interface was implemented using the Qt5.4 \cite{Qt} framework. 
\begin{figure}
\centering
\includegraphics[scale=0.4]{Pictures/CADO_mainWindowParameters.png}
\caption{CADO User Interface. Sample parameters for the cantilever example}
\label{fig:mainWindowParameters}
\end{figure}
The interface allows to enter all necessary files and parameters without typing them into the command line. In particular, user chooses only the main \textit{.step} file and then, if all other files were named according to the naming convention (see sec. \ref{sec: CADToVoxels}), user just has to check the checkbox for specifying fixtures or the optimization domain (see fig. \ref{fig:mainWindowParameters}).

All necessary input parameters for the topology optimization can be entered through the input fields:
\begin{itemize}
\item Force Scaling - parameter for the force scaling (see sec. \ref{sec: CADToVoxels}).
\item Resolution - voxel size for the voxelization. The entered parameter is then taken into the power of 2.
\item Volume Fraction Limit - the fraction of the volume to be kept after the topology optimization process by ToPy (see sec. \ref{sec:ToPy}).
\end{itemize}
\todo{Do we subtract one from it now or not???}
All necessary input parameters for the surface fitting (see sec. \ref{sec:LSQfitting}) can be entered through the input fields:
\begin{itemize}
\item Smoothing - parameter for the fairness functional (see sec. \ref{sec:NURBS})
\item Coarsening - the number of coarsening steps in Dual Contouring.
\end{itemize}

\todo{Explain coarsening somewhere in the report!}
After all parameters are specified, the pipeline can  be executed without any interaction with the user. After the process has finished, there is an option to lunch FreeCAD directly from the GUI  and view the results.


