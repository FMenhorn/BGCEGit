\subsection{Parametrization of Datapoints}
\label{ssec:parametrization}
In addition to the reconstructed surface we need the following information for the least squares fit: 
\begin{itemize}
\item Which \ac{NURBS}-patch does each datapoint of the reconstructed surface belong to?
\item Which parameters $u,v$ does the datapoint have on this patch?
\end{itemize}
\subsubsection{Two-scale Dual Contouring}
Before we can distribute datapoints to \ac{NURBS}-patches, we first have to find out how these patches look like. Since we want to have as few patches as possible we do not simply turn every \ac{quad} from the surface reconstruction into a patch. Instead  we try to find a surface with as few \acp{quad} as possible, while keeping the same topology as our initially reconstructed surface. This coarse surface will be assumed to be the patch distribution for the later steps.


Therefore, in our algorithm we are reconstructing the surface on two different scales: a coarse and a fine scale. The coarse scale data is deduced from the fine scale data by recursively applying the following algorithm:
\begin{enumerate}
\item Combine sets of eight connected voxels with egde-length $a$ into a coarse voxel with edgelength $2a$ (see \autoref{sfig:coarsenA}).
\item Decide whether the new voxel resembles an inside ($=1$) or outside ($=0$) voxel, by taking the mean value of the contributing eight voxels. If mean value is above a certain threshold $t$, the new voxel is considered as an inside voxel. We picked $t=\frac{1}{8}$, i.e. if at least two voxels out of eight are inside voxels, the resulting voxel is considered being an inside voxel.
\item For the resolution of non-manifold edges additional coarse voxel grids are generated, that are shifted by edgelength $a$ (see \autoref{sfig:coarsenB}).
\end{enumerate}
\begin{figure}
\begin{subfigure}{.45\textwidth}
\begin{center}
\includegraphics[scale=1]{Pictures/tikzCoarsen/coarsen.pdf}
\end{center}
\subcaption{Recursively sets of 8 (4 in 2D) fine voxels (blue) are combined to form one coarse voxel (red).}
\label{sfig:coarsenA}
\end{subfigure}\hfill
\begin{subfigure}{.45\textwidth}
\begin{center}
\includegraphics[scale=1]{Pictures/tikzCoarsen/coarsenMani.pdf}
\end{center}
\subcaption{In a non-aligned grid addition voxels (yellow) are introduced, that are used for non-manifold resolution on the coarse scale.}
\label{sfig:coarsenB}
\end{subfigure}
\caption{Coarsening scheme applied for \emph{Two-scale \acl{DC}}.}
\label{fig:coarsen}
\end{figure}
One iteration of this coarsening scheme is referred to as one \emph{coarsening step}, applying this scheme recursively allows higher coarsening and results in multiple coarsening steps.

In the following we apply \ac{DC} to both datasets and obtain two different reconstructed surfaces.
The coarse scale surface is used as a patch distribution; from the fine scale we obtain our datapoints to which the \ac{NURBS} will be fitted to. We call this approach \emph{Two-scale \acl{DC}} (see \autoref{fig:TwoScale}).
Of course this simple approach comes with drawbacks:
\begin{itemize}
\item We cannot guarantee that the coarse and the fine scale have exactly the same topology. Topological details of the fine scale, which do not exist on the coarse scale, are lost.
\item Both resolutions have to be chosen manually, since we do not have a criterion for evaluating the quality of the coarse scale surface reconstruction.
\end{itemize}
These drawbacks are especially bad for complex surfaces -- like the output of topology optimization -- where the topological details mentioned above are not an exception, but the default case. Possible solutions to these problems by improving our surface extraction scheme are considered in \autoref{sec:Future}.
\subsubsection{Projection of Datapoints onto Quads}
Now that we have constructed a \ac{NURBS}-patch distribution we can estimate the parametrization of the datapoints on the fine scale by projecting the datapoints onto the patches: For this procedure we first have to find out onto which one of the patches one particular datapoint will be projected. Then we have to do the actual projection of the datapoint onto the quad: 
\begin{itemize}
\item The first part can be done by simply measuring the distance from the datapoint to the centroid of each patch and deciding to project onto the patch with the smallest distance. 

\item For the latter we want the whole \ac{quad} to be parametrized on $u,v\in\left[0,1\right]^2$. We could either use bilinear interpolation for representing the quad and solve a non-linear minimization problem for each projection, or first approximate the quad with its least squares fit plane and then do a projection onto this plane by applying a simple basis transformation\footnote{This basis transformation can be computed in a very cheap way, by computing the QR-decomposition for the basis of each patch only once and applying it for each datapoint projected onto this patch.}. In our algorithm we used the second approach since the approximation of the original \acp{quad} via their least-squares fit planes introduces only a small error and significantly simplifies the projection onto the \ac{quad}.
\end{itemize}
Finally the projection might lead to parameters $u,v\not\in\left[0,1\right]^2$ for some of the datapoints. As these are not directly above a \ac{quad}, they might lead to inconsistencies in where the datapoints of a quad belong to, especially along interfaces of quads, where points on different quads might lie overlapping. To simplify the treatment and only use consistent points, these outside points are therefore omitted.
\tododone[author=Benni]{{@} Erik: really?}\tododone[author=Erik]{{@} Benni: no. Changed stuff}


After having completed all these steps we obtain the following data for the subsequent steps of our algorithm:
\begin{itemize}
\item a coarse surface delivering the topology for our \ac{NURBS}-patches in Peters' Scheme
\item a set of datapoints from the fine scale with coordinates $\left(x,y,z\right)^T$ and parameters $\left(u,v\right)\in\left[0,1\right]^2$, where each datapoint is associated with a \ac{NURBS}-patch. 
\end{itemize}
For a visualization of our algorithm in 2D see \autoref{fig:TwoScale}.

\begin{figure}
\begin{center}
\includegraphics[width=.5 \textwidth]{Pictures/SurfaceReconstruction/TwoScale}
\caption{Twoscale \acl{DC} with a coarse surface reconstruction (red) and a fine one (blue). The datapoints from the fine scale are projected onto the edges from the coarse scale (black lines).}
\label{fig:TwoScale}
\end{center}
\end{figure}
\begin{comment}
\subsubsection{Parameter estimation}
\todo[inline]{explain different strategies, comparison of final results would be great (could also be part of third milestone?)}
\todo[inline]{show possible problems}
\end{comment}
