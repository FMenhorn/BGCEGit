\section{From NURBS to Standardized CAD File Format}
\tododone[inline,author=Benni]{Should be rather short. Just explain that we use FreeCad Python interface. Severin finished this section, complain to me! =)}
In order to create standardized CAD files from previously computed B-Spline control points, the scripting functionality offered by FreeCAD is employed. Almost all functions of FreeCAD can be called using a python script, which allows to utilize FreeCAD functions within the automatized tool-chain. \cite{FreeCAD}

%installation path
Thus, the implemented python script is structured as follows: 
\begin{enumerate}
\item The installation path is specified and FreeCAD modules are imported separately
\item A new document is opened and B-Spline patches are consecutively created according to control points from Peters' scheme
\item  The object is reoriented to revert coordinate changes imposed by Topy (see section \ref{sec: ToPyInputConstruction})
\item Geometric constraints are enforced with boolean operations in FreeCAD  
\item The active object is exported as step file.
\end{enumerate} 




