\subsection{Voxelization}
\label{sec: Voxelization}
%what is voxelisation

As pointed out in section \ref{sec:TopOpt}, a very common formulation of domains for topology optimization is through specifying regions as either filled or empty. The minimum compliance problem is then solved on a discretized grid; the most common one is a volume raster in the form of cubes, or so called \emph{voxels}. Thus, the next step is to render the geometry with a 3D raster of voxels.

From the face extraction pipeline, the geometry shape and faces for each boundary condition type are stored in OpenCascade through the internal data type \lstinline|TopoDSShape|. The \lstinline|voxelise| function is called internally for the complete geometry and each face separately (see \autoref{fig:umlCADToVoxel}) since boundary conditions may consist of more than one face. The voxelisation is then performed as follows: 
\begin{enumerate}
\item In order to combine the 3D voxel raster consistently, a bounding box is introduced. This allows keeping the coordinate system uniform, making voxel numbers consistent between different voxelization types (faces and shapes). 
\item In each dimension, $2^n \cdot l_d$ voxels are created, where $n$ is the user specified refinement level and $l_d$ is the size of the bounding box in the respective dimension $d$.
\item Voxelization is performed with the OpenCascade  \lstinline|Voxel_FastConverter.hxx| class creating a \lstinline|VoxelShape|.
\end{enumerate}

Consequentially, a 3D boolean voxel raster is created for each type.
