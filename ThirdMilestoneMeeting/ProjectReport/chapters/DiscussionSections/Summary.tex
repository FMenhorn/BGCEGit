\section{In a Nutshell:\ \acl{CADTopOpt}}
\label{sec:nutshell}
\todourgent[author=Benni]{@ Erik: please proofread! Especially discussion or results.}
\todourgent[author=erik]{@Erik:maybe talk a bit more about summary of the report?}
In this report, we present CADO, Computer Aided Design Optimizer, which is the result of this project. In summary, CADO works as a fully integrated tool-chain from the CAD input file to an optimized CAD file. First, the input geometry undergoes voxelization using OpenCASCADE to ensure compatibility with the topology optimizer. Second, the topology is optimized by employing the open-source tool ToPy \cite{ToPy}.  
Next, a two-stage Dual Contouring surface reconstruction scheme is executed on the output of topology optimization. This gives us coarse parametrization patches and fine vertices as output.
A B-Spline surface is then fitted through this data by a least-square approach using control points described by the Peters' scheme \cite{peters1992constructing} in order to ensure continuous and smooth surfaces. Lastly, a FreeCAD macro script performs boolean operations to enforce geometric constraints and exports the geometry to a standardized CAD file \cite{FreeCAD}.
Note that intermediate steps can be replaced in the future with more suitable solutions. 

The functionality of the tool was tested through three test cases described in \autoref{sec:tests}. We made the following observations:
\begin{itemize}
\item The design problem could be easily formulated in the form of CAD files.
\item The CAD files were parsed to construct the topology optimization problem. Only small problems (Bridge, Cantilever) could be handled using ToPy. For bigger problems (GE Jet Engine Bracket) ToPy was not sufficient.
\item Smooth NURBS surfaces of arbitrary topology were successfully reconstructed from the optimization solution, albeit with a high number of patches. Deformations of the resulting NURBS surfaces were observed ("dents" in the bridge). The reason for this is limited accuracy in the projection scheme described in \autoref{sssec:projection}. Additionally, conservation of topological features was not guaranteed (GE Jet Engine Bracket).
\item Finally, after post-processing for geometry constraints, the result was exported as a standard CAD \emph{.step} file.
\end{itemize}
CADO is a strong proof of concept for CAD integrated topology optimization. It solved the proposed test scenarios to a qualitative level of satisfaction. Nevertheless, CADO's maturity to a full-fledged software package for engineering problems requires further improvements and additions.