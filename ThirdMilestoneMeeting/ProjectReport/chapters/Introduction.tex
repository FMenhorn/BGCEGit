\chapter{Introduction}
\label{chapter:Introduction}
In this chapter, we develop the the motivation for the project and provide a short description of the problem task, together with a brief introduction to topology optimization and the project structure.
\section{Motivation}
% Struggle designer engineer 
A common problem in product design is to create a functioning structure using as little material as possible. Three decades ago, engineering design versions were drawn, prototypes created and experimental test performed. Nowadays, the field of topology optimization simplifies this process and stands as a powerful tool in engineering and design.

Topology optimization tackles the problem of material distribution in a structure in order to fulfil certain target loads. Several topology optimization open-source tools exist that are ready to use; however, it is still a challenge to incorporate these tools smoothly in the design process. The idea of this project is to allow these tools to work starting directly from \acf{CAD} files and to transfer the resulting mesh-based solution back to the \ac{CAD} world. Unfortunately, at the moment, there is no open-source solution for the conversion of mesh-based geometry to the spline-based \ac{CAD} format. The common approach of converting each triangle of a mesh geometry directly into \ac{CAD} format results in enormous file sizes. One of the biggest challenges of this project is thus to develop a conversion tool that feasibly provides a useful \ac{CAD}-representation of the optimized surface.

%Concepts are described in the Background theory section

%\section{Important concepts}
%
%\subsection{Computer Aided Design --- CAD}
%
%\subsection{Topology Optimization}

\section{Project Structure}
\todointern{Erik: review this part}
The aim of the project was to provide a tool that allows to utilize topology optimization without leaving the CAD-framework. Therefore, the main goals of the project were as follows:
\begin{itemize}
\item Implementation of a topology optimization framework (by using and extending available open source libraries) that accepts geometry in CAD format.
\item Development of a flexible tool for conversion of an optimized surface back to the CAD format.
\end{itemize}

The duration of the project had been set to 10 months. Hence, it was divided into 4 phases:\\

\textbf{Phase 1:} Getting familiar with the topic and agreement on the project specification.\\

\textbf{Phase 2:} Implementing the first part of the pipeline (Topology Optimization from CAD surface using existing tools); investigating the tools and algorithms available for the conversion of the geometry generated after topology optimization back to CAD format (later referred as \emph{NURBS fitting pipeline}); prototyping (using MATLAB) and evaluating results.\\

\textbf{Phase 3:} Implementing the prototypes developed on the previous stage, using non-proprietary languages; extending the NURBS fitting pipeline to more complex cases; finalising the first part of the pipeline.\\

\textbf{Phase 4:} Implementing the extended NURBS fitting pipeline; integrating with the topology optimization part and delivering the final product to costumer; providing a user-guide and an installation guide
