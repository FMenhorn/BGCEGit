\subsection{Fairness Functional}%maybe change title
\label{subsec:fairnessthry}
\todointernal[inline, author=Erik]{on reread: come up with nicer-looking variable names}
Aside from recognising discontinuities in the first derivative (kinks), the human eye is usually able to recognise large values in curvature, that is, properties depending on the second derivatives of a surface. For example, we can have a surface that is entirely continous in the first derivative of the surface, that also has large wiggles and sharp turns, that on a small scale are smooth, but do not seem so on the large scale. %maybe example of a gaussian with very small variance? 
\todointernal[author=Erik]{mention that it's called fairness because they're nice to look at}
To give a measure for this, one can for example use the bending energy of a thin plate, which depends on the squared magnitude of the second derivatives. In the special case of a rectangular patch, parametrized by $u,v\in \left[0,1\right]$, we can easily integrate this over the whole area. The energy can then be expressed as a functional of $\vec{S}$,

\begin{equation}
\label{eq:fairness}
E_{patch}\left[\vec{S}\right] = 
\int\limits_0^1 \int\limits_0^1
\left(
\left(
\frac{\partial^2}{\partial u^2 }\vec{S}(u,v)
\right)^2
+ 2 
\left(
\frac{\partial^2}{\partial u\partial v} \vec{S}(u,v)
\right)^2
+
\left(
\frac{\partial^2}{\partial v^2}\vec{S}(u,v)
\right)^2
\right)
\text{d}u\text{d}v
\end{equation}
One case in which this equation can be simplified is when we have explicit definitions of the surface. For example, for the rectangular \Bez tensor product surface curves in \autoref{subsec:paracurves}, we express the surface as a weighted sum of polynomials in $u$ and $v$ (see \autoref{eq:bezsurface} for a reminder). For a \Bez surface of order $N$, with $N+1$ points $\vec{p}_{i,j}$ in each direction, we can thus reexpress \autoref{eq:fairness} as coefficients times a dot product 
\begin{equation}
\label{eq:faireqsum}
E_{fair} =  \sum\limits_{i_1,j_1=1}^{N+1} \sum\limits_{i_2,j_2=1}^{N+1} \left( c_{(i_1,j_1),(i_2,j_2)}^{uu} + 2c_{(i_1,j_1),(i_2,j_2)}^{uv} + c_{(i_1,j_1),(i_2,j_2)}^{vv} \right) \vec{p}_{i_1,j_1} \cdot\vec{p}_{i_2 ,j_2} 
\end{equation} 
where the coeficcients $c_{(i_1,j_1),(i_2,j_2)}^{st}$ are obtained by of multiplying out the squares in the energy functional, inserting the definitions of the Bernstein polynimials $b_{i}^N(u)$:
\begin{equation}
c_{(i_1,j_1),(i_2,j_2)}^{st} = 
\int\limits_0^1 \int\limits_0^1 \frac{\partial^2}{\partial s \partial t} \left[b_{i_1}^N(s) b_{j_1}^N(t)\right]\frac{\partial^2}{\partial s \partial t} \left[ b_{i_2}^N(s) b_{j_2}^N(t)\right]
\text{d}s\text{d}t
\end{equation}
\todointern[author=Erik]{maybe move this following part to LeastSquares?}
Although this is already very much simpler to evaluate, given one already has the coefficients $c_{(i_1,j_1),(i_2,j_2)}$, we can do a further reformulation by reindexing $(i,j)$ to $k$ where $k$ goes from $1$ to $(N+1)\times(N+1)$. By forming a vector $P_{patch}$ containg all the points as 3-dimensional entries, we can then reexpress \autoref{eq:faireqsum} as a vector-matrix-vector product:
\begin{align*}
\phantom{=}& \sum\limits_{i_1,j_1=1}^{N+1} \sum\limits_{i_2,j_2=1}^{N+1} c_{(i_1,j_1),(i_2,j_2)} \vec{p}_{i_1,j_1} \cdot\vec{p}_{i_2 ,j_2} =
\\
=& \sum\limits_{k_1=1}^{(N+1)^2} \sum\limits_{k_2=1}^{(N+1)^2} c_{k_1,k_2} \vec{p}_{k_1} \cdot\vec{p}_{k_2} =
\\
=& P_{patch}^T C_{patch}^{fair} P_{patch}
\end{align*}
Now, since we know the original expression of $E_{fair}$ and the contribution stemming from each point $\vec{p}_{i,j}$, we know the matrix $C_{patch}^{fair}$ to be positive semi-definite and symmetric. Hence, it should be diagonalizable by an orthogonal matrix, and have only nonnegative eigenvalues. Thus, expressing $C_{patch}^{fair} = S^T \Lambda S$, we end up with:
\begin{align}
E_{fair} =& P_{patch}^T C_{patch}^{fair} P_{patch} =\nonumber
\\
=&  P_{patch}^T S^T \Lambda S P_{patch} = \nonumber
\\
=& P_{patch}^T S^T \Lambda^{\frac{T}{2}}\Lambda^{\frac{1}{2}} S P_{patch} = \nonumber
\\
=& \left[\Lambda^{\frac{1}{2}} S P_{patch}\right]^T\left[\Lambda^{\frac{1}{2}} S P_{patch}\right] =\nonumber
\\
=& \norm{\Lambda^{\frac{1}{2}} S P_{patch}}^2
\label{eq:fairnessnorm}
\end{align}
which allows us to express and calculate the thin plate energy using a matrix-vector product and a squared norm.

Since surfaces with low curvatures and therefore softer curves are typically considered more pleasing to the eye, the functional in \autoref{eq:fairness} is typically named \emph{fairness functional}.\todointern[author=Erik]{move this further up}
%%plotline: 
%as well as smoothness, curvature is important
%for quadratic parametric patch, can express curvature by thin-plate energy functional as was done in [eck-hoppe]
%we notice here that for Bezier curves, we express the \vec{S} surface as polynomials*surface points, which we can integrate analytically
%AND: this gives us a quadratic polynomial in the control points of the patch, with a lot of coefficients
%we express this as the norm in a matrix-vector thingy:
% <from here in LSQ section> and this we can then find the eigenvectors for
% which let us express this as a sum of squares of coefficients*bezier points, (just like we wanted.)

%
%\begin{align}
%\vec{d}_k = & \sum\limits_{i,j=1}^{N,M} \c(u_k,v_k)_{i,j} \lsControlPointi{i,j}
%\\=& C_k (u_k,v_k)P
%\end{align}
%\begin{align}
%\vec{p}_{p} &= \sum\limits_{l} c^{PS}_{p,l} \vec{v} 
%\\
%&=C_k^{PS}V
%\end{align}
%
%
%
%\begin{equation}
%\label{eq:fairness-sum}
%E_{fair} = \sum\limits_{\text{all quads}}E_{fair,quad}
%\end{equation}


