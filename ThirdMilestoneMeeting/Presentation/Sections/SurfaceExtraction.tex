
\subsubsection{Dual Contouring}

\begin{frame}
\frametitle{From Voxel to Mesh Geometry\footnote{\textbf{Tao Ju, Frank Losasso, Scott Schaefer, Joe Warren.} "Dual contouring of hermite data"}}
\vspace{-0.3cm}
\begin{center}
	\textbf{Goal:} Extract isosurface from voxel information	
\end{center}
\vspace{-0.5cm}	
\begin{minipage}[t]{0.4\linewidth}
	\begin{block}{Marching Cubes}
		\begin{enumerate}
			\item<2-> Identify boundary voxels
			\item<3-> Locate sign changes on cube edges
			\item<4-> Reconstruct surface
		\end{enumerate}
	\end{block}
	\only<1>{
		\begin{figure}	
			\centering
			\includegraphics[width=.5\textwidth]{Pictures/DC/MC1.png}
		\end{figure}}
	\only<2>{
		\begin{figure}	
			\centering
			\includegraphics[width=.5\textwidth]{Pictures/DC/MC2.png} 
		\end{figure}}
	\only<3>{
		\begin{figure}	
			\centering
			\includegraphics[width=.5\textwidth]{Pictures/DC/MC3.png} 
		\end{figure}}
	\only<4->{
		\begin{figure}	
			\centering
			\includegraphics[width=.5\textwidth]{Pictures/DC/MC4.png} 
			\end{figure}}
\end{minipage}%
\hfill%
\begin{minipage}[t]{0.4\linewidth}
\begin{block}<5->{Dual Contouring}
	\begin{enumerate}
		\item<6-> Identify boundary voxels
		\item<7-> Locate position inside boundary voxel
		\item<8-> Reconstruct surface
	\end{enumerate}
\end{block}
\only<5>{
		\begin{figure}	
		\centering
		\includegraphics[width=.5\textwidth]{Pictures/DC/MC1.png} 
		\end{figure} }
\only<6>{
		\begin{figure}	
		\centering
		\includegraphics[width=.5\textwidth]{Pictures/DC/MC2.png} 
		\end{figure}}
\only<7>{
		\begin{figure}	
		\centering
		\includegraphics[width=.5\textwidth]{Pictures/DC/DC3.png} 
		\end{figure}}
\only<8->{
		\begin{figure}	
		\centering
		\includegraphics[width=.5\textwidth]{Pictures/DC/DC4.png} 
		\end{figure}}
\end{minipage}
\vspace{0.4cm}

\pause
\pause
\pause
\pause
\pause
\pause
\pause
\pause
\vspace{-.1cm}
\begin{minipage}[c]{0.5\linewidth}
\hspace{1cm}$\rightarrow Triangles$
\end{minipage}%
\hfill%
\begin{minipage}[c]{0.5\linewidth}
\begin{center}
	{$\rightarrow Quads$}
\end{center}
\end{minipage}
\end{frame}


\begin{frame}
\frametitle{Two-grid Dual Contouring}
\begin{overlayarea}{\textwidth}{0.9\textheight}
	\begin{minipage}{0.45\textwidth}
	\begin{block}{\centering Coarse grid}
	\vspace{-0.5cm}
	\begin{figure}
	\includegraphics[scale=0.35]{Pictures/DC/DC_1_Coarse.pdf}
	\end{figure}
	\begin{itemize}
	\item Coarse quads used in \textcolor{red}{parametrization}
	\end{itemize}
	\end{block}
	\end{minipage}
	\hfill%
	\begin{minipage}{0.45\textwidth}
	\begin{block}{\centering Fine grid}
	\vspace{-0.5cm}
	\begin{figure}
	\includegraphics[scale=0.35]{Pictures/DC/DC_1_Fine.pdf}
	\end{figure}
	\begin{itemize}
	\item Fine vertices used for \textcolor{red}{projection}
	\end{itemize}
	\end{block}
	\end{minipage}
\end{overlayarea}
\end{frame}

%\subsection{B--Spline}

\subsection{Projection and Parametrization}
\begin{frame}{Projection and Parametrization}
%\framesubtitle{Least square fitting}
\begin{overlayarea}{\textwidth}{.9 \textheight}
%\begin{minipage}{0.45\textwidth}
\begin{enumerate}
\visible<1->{\item Use coarse quad from Dual Contouring}
\visible<2->{\item Project grid points from fine grid onto closest plane}
\visible<4->{\item Find corresponding parameters for B-Spline surface $\left[u,v\right] \in \left[0,1\right]^2$}
\visible<4->{\alert<4->{\item[$\Rightarrow$] Peter's scheme}}
\end{enumerate}
%\end{overlayarea}
%\begin{overlayarea}{\textwidth}{.85 \textheight}
%\end{minipage}
\vspace{-0.5cm}
%\begin{columns}
%\column{.35\textwidth}
%\begin{overlayarea}{\textwidth}{\textheight}
\begin{figure}
\visible<1->{
\tdplotsetmaincoords{60}{110}
\begin{tikzpicture}[scale = 1.5,tdplot_main_coords]
\coordinate (O) at (-1,-1,0);
\coordinate[dot] (A) at (0,0,0);
\coordinate[dot] (B) at (1,0,0);
\coordinate[dot] (C) at (1.2,1.5,0);
\coordinate[dot] (D) at (0,1,0);
\visible<3->{
\coordinate[dot] (E) at (1.4,2.0,-0.5);
\coordinate[dot] (F) at (0,1.7,-0.8);
\draw[thick] (D) -- (F) -- (E) -- (C);}

\coordinate (Q1) at (.5,.4,0);
\coordinate (Q2) at (1,1,0);
\coordinate (Q3) at (0.2,0.2,0);
\coordinate (Q4) at ($.5*(C)+.5*(F)$);

\coordinate (P1) at (.5,.4,1);
\coordinate (P2) at (1,1,1);
\coordinate (P3) at ($(Q3)+(0,0,2)$);
\coordinate (P4) at ($(Q4)+(0,.5,2)$);


%% coordinate system
%\draw[thick,->] (O) -- ($(O)+(.5,0,0)$) node[anchor=north east]{$x$};
%\draw[thick,->] (O) -- ($(O)+(0,.5,0)$) node[anchor=north west]{$y$};
%\draw[thick,->] (O) -- ($(O)+(0,0,.5)$) node[anchor=south]{$z$};

\draw[thick] (A) -- (B) -- (C) -- (D) -- (A);

\draw (P1) node[thick,cross,red,label = {$P_1$}] {};
\draw (P2) node[thick,cross,red,label = {$P_2$}] {};
\draw (P3) node[thick,cross,red,label = {$P_3$}] {};
\visible<2->{
\draw (Q1) node[thick,cross,red] {};
\draw (Q2) node[thick,cross,red] {};
\draw (Q3) node[thick,cross,red] {};
\draw[red,dashed] (P1) -- (Q1);
\draw[red,dashed] (P2) -- (Q2);
\draw[red,dashed] (P3) -- (Q3);
}

\visible<3->{
\draw (P4) node[thick,cross,red,label = {$P_4$}] {};
\draw[red,dashed] (P4) -- (Q4);
\draw (Q4) node[thick,cross,red] {};
}
\draw (A) node[]{};
\draw (B) node[]{};
\draw (C) node[]{};
\draw (D) node[]{};
\draw (E) node[]{};
\draw (F) node[]{};
\visible<4->{
\draw[blue,thick,->] (B) -- ($.5*(A)+.5*(B)$) node[anchor=east]{\textcolor{blue}{$u$}};
\draw[blue,thick,->] (B) -- ($.75*(B)+.25*(C)$) node[anchor=north]{\textcolor{blue}{$v$}};
}
\end{tikzpicture}
}
\end{figure}
%\end{overlayarea}

%\column{.5\textwidth}
%\begin{overlayarea}{\textwidth}{\textheight}
%\only<3->{
%\begin{block}{Problem:}
%\begin{itemize}
%\item Fit B-Spline surface, that is C0 and C1 continuous on the borders
%\end{itemize}
%\end{block}
%
%\begin{block}{Solution:}
%\begin{enumerate}
%\item Method: Peter's scheme
%\item Solve (coupled) global system of equations
%\end{enumerate}
%\end{block}
%
%}
%\end{overlayarea}
%\end{columns}
\end{overlayarea}
\end{frame}

%\begin{frame}
%
%	\frametitle{Projection and Parametrization}
%	
%	\begin{itemize}
%	\item Points from finer grid are projected to quads of the coarser grid 
%	\item Parameters \textit{u} and \textit{v} are found for each quad
%	\item This information is needed for the algorithms in the last part of the pipeline
%	\end{itemize}
%	\begin{figure}
%	\includegraphics[scale=0.35]{Pictures/DC/DC_2.pdf}
%	\end{figure}
%	
%\end{frame}





