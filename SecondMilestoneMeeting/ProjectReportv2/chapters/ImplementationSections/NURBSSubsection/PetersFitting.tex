\subsection{Applying Peters' Scheme}
In this section we will describe:
\begin{itemize}
\item That we apply the theory in section \autoref{sec:NURBS} about Peters' Scheme to fit datapoints from Dual Contouring in \autoref{sec:DC}\todointern{fix this ref} and the coarse quads and the parameters on them to obtain a $G^1$ surface
\item That we implemented this as Python classes and structures after MATLAB prototyping, \autoref{fig:fittingStructures} for reference
\item How they work algorithmically, \textit{(whenever they work)}
\item What libraries we used to do the least-squares an plotting an stuff
\end{itemize}

\todourgent[inline]{does anybody have a clue why there are those lines at the end of each chapter? In my opinion they are quite ugly...}

\begin{figure}
\missingfigure{Graphical descirption of all those different points with fitting steps in Peters' Scheme. , scrap that, let's do a class diagram instead. No wait! Prototype pictures could be shown here, and some pictures from different funny fitted shapes!}
\label{fig:fittingStructures}
\caption{The \emph{results} of Peters' Scheme fitting. They're really good.}
\end{figure}