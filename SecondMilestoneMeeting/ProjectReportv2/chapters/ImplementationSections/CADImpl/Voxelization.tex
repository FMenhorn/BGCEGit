\subsection{Voxelization}
\label{sec: Voxelization}

\tododone[inline]{Severin: Describe the voxelization process. Explain concept of refinement.}
%what is voxelisation

As it is pointed out in section \ref{sec:TopOpt}, a very common formulation of topology optimization deals with regions that are specified as filled or empty. The minimum compliance problem is then solved on a discretized grid; the most common one is a volume raster in the form of cubes - so called voxels. Since ToPy requires a voxel grid in their input format, the next step is to render the geometry with a 3D raster of voxels.

As described in the previous section, the geometry shape and faces for each boundary condition type, are stored in OpenCascade through the internal data type \lstinline|TopoDSShape|. As one can see in the UML Diagram \ref{fig: umlCADToVoxel} the \lstinline|voxelise| function is called internally by \lstinline|CADTOVoxel| -- for the shape and each faces separately. The voxelisation is then performed as follows: 
\begin{enumerate}
\item In order to combine the 3D voxel raster consistently, a bounding box is introduced. As a consequence the same coordinate system is used and voxel numbers are consistent between different voxelization types (faces and shapes). 
\item In each dimension $2^n*l_d$ voxels are created, where $n$ is the user specified refinement level and $l_d$ is the size of the bounding box in the respective dimension $d$
\item Voxelization is performed with the OpenCascade  \lstinline|Voxel_FastConverter.hxx| class creating a \lstinline|VoxelShape|
\end{enumerate}
\tododone[inline]{OpenCascade TopoBoolDS shape}

Consequentially, a 3D boolean voxel raster is created for each type -- since boundary conditions may consist of more than one face, they are voxelized separately and lists of \lstinline|VoxelShape| are created. 
