\section{From \acs{CAD} to Voxels}

\todo[inline]{Include a super-diagram of the classes being used}
\label{sec: CADToVoxels}
One of the hurdles with most state-of-the-art open source topology optimization tools is their input format, where many of them (including ToPy, our topology optimizer of choice) require input to be specified as a 3-dimensional voxel grid. Presence (or absence) of material in these voxels is defined by a boolean variable, and boundary conditions are imposed on the appropriate locations. An example of voxelized data can be seen in \autoref{fig: voxelOpenCascade}. %Since there is a variety of toolboxes available which are able to perform a voxelization of common CAD input files, we did not implement one of our own but rather adapted one of the open source tools. %This section describes how we overcame this hurdle of converting CAD representations to voxelized input.

\subsection{Specification of Boundary Conditions for the Input Geometry}
\label{sec: GeomCreation}

\todo[inline]{Saumi: Here we describe what one needs to do in FreeCAD to set the loads, fixtures, and passive elements for the geometry. We also instruct the user to save the geometry as both IGES and STEP, and explain the reason for this.}


\subsection{Face Extraction and Categorization}
\label{sec: FaceExtraction}

%\tododone[inline]{}
Apart from using CADO directly, one can also call the face extraction step manually. For that, once the user has created their input geometry and saved it as STEP and IGES files, they can begin the execution by invoking the bash script \lstinline|CADTopOp.sh| located in \lstinline|/CADORoot/CADO/CPP|. For example, for a case located in directory \lstinline|/newuser/CADRoot/TestCase/| with file name \lstinline|geom|, force scaling factor of 240.19 and a refinement level of 2, the appropriate command to invoke the script from its folder would be: \\

\lstinline|$ ./CADTopOp.sh /CADRoot/TestCase/ geom 240.19 2| \\

The bash script basically invokes \lstinline|CADToVoxel| that parses the geometry, sorts out the faces based on their type, voxelizes the assembly, and writes the input for the topology optimizer. The script then calls the topology optimizer with this file as input. Once optimised, the script requests an extracted surface from the density voxel grid from the dual contouring algorithm. 

%It is worth noting, that as of the second milestone an integrated pipeling is available from the geometry input to the surface extraction through dual contouring. Connection with the next part of the pipeline is the objective for the third phase.

Here is an explanation of how the faces are extracted from the CAD input and how they are categorised based on the color:

\begin{enumerate}
	\item An instance of \lstinline|Reader| is created with the CAD source directory and file name as input. \lstinline|Reader| wraps the OpenCascade classes for reading STEP and IGES files into a single class, reads the two files, and holds them in two handles. Also \lstinline|Reader| instances are created for the non-changing domain and load file input.
	\item The \lstinline|ColorHandler| class takes over the handles from \lstinline|Reader|. \lstinline|ColorHandler| provides methods, each of which returns a list of faces (see \autoref{fig:umlCADToVoxel}). Depending on which method is called the returned list contains groups of fixtures, loads, passive faces, or all faces of the body.
	\item Each of the methods mentioned above internally calls the hidden function \lstinline|findColoredFaces()|. This method takes as input a color, and returns all faces that match it. It also takes as input a boolean variable \lstinline|isLoadSeeked| - if true, then the function returns all faces with load on them, and also a vector of the corresponding loads.
	\item The load vector is then scaled with respect to the scaling factor provided as input by the user.
\end{enumerate}

Once these face lists are available, they can be voxelized.


\subsection{Voxelization}
\label{sec: Voxelization}

\todointern[inline]{Severin: Describe the voxelization process. Explain concept of refinement.}
%what is voxelisation
As pointed out in section \ref{sec:ToPy} a very common formulation of topology optimization deals with regions that are specified as filled or empty. The minimum compliance problem is then solved on a discretized grid; the most common one is a volume raster in the form of cubes - so called voxels. Since ToPy requires a voxel grid in their input format, the next step is to render the geometry with a 3D raster of voxels.  

As described in the previous section, the geometry shape and faces for each boundary condition type, are stored in OpenCascade through the internal data type \lstinline|TopoDSShape|. As one can see in the UML Diagram \ref{fig: umlCADToVoxel} the \lstinline|voxelise| function is called internally by \lstinline|CADTOVoxel| -- for the shape and each faces separately. The voxelisation is then performed as follows: 
\begin{enumerate}
\item In order to combine the 3D voxel raster consistently, a bounding box is introduced.
\item In each dimension $2^n*l_d$ voxels are created, where $n$ is the user specified refinement level and $l_d$ is the size of the bounding box in the respective dimension $d$
\item Voxelization is performed with the OpenCascade  \lstinline|Voxel_FastConverter.hxx| class creating a \lstinline|VoxelShape|
\end{enumerate}
\todointern[inline]{OpenCascade TopoBoolDS shape}



\subsection{Construction of ToPy Input File}
\label{sec: ToPyInputConstruction}

\todo[inline]{Severin: Explain the difference in the coordinate system of ToPy. Explain the mirroring of directions being done in the ToPyWriter.}


\begin{figure}
\centering
  \includegraphics[scale=0.3]{Pictures/CADToVoxel/voxels_wp_image005.png}
\caption{A shape and its voxel representation. \emph{Leftmost picture}: The original parametrized shape. \emph{Rightmost picture}: The voxel representation. Picture from OpenCascade \cite{OpenCascade}.}
\label{fig: voxelOpenCascade}
\end{figure}

\begin{figure}
\centering
\begin{subfigure}{
  \includegraphics[width=.2\linewidth]{Pictures/STLToVoxels/Star_STL.png}}
\end{subfigure}
\begin{subfigure}{
  \includegraphics[width=.2\linewidth]{Pictures/STLToVoxels/Star_VTK_Trans.png}}
\end{subfigure}
\caption{The STL geometry of a star (left) and its voxelized form (right) obtained via the CVMLCPP voxelizer, visualized by Paraview \cite{Paraview}.}
\label{fig: voxelizerStar}
\end{figure}

In terms of implementation, the only thing required is the installation of the CVMLCPP library. The voxelizer is then included as a callable binary, that takes STL-file input (\autoref{subsub:STL}) and converts it to a .dat binary file with dimension and voxel information, with specifiable voxel size. An example result of using the voxelizer is shown in \autoref{fig: voxelizerStar}.
 %In the end, to voxelize a file given as .stl-input, we could just call 
%\begin{lstlisting}
%~/Path/To/CVMLCPP/bin/voxelize ./<stl_file>.stl <voxelSize>
%\end{lstlisting}
%where $\mathtt{<voxelSize>}$ is an integer declaring the size of a voxel. 
