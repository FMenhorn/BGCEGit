\subsection{Dual marching methods}
So far we have introduces \ac{MC} and \ac{DC} and discovered that both methods have certain drawbacks:
\begin{itemize}
\item \ac{MC} always produces manifold surfaces and resolves ambiguities correctly, while \ac{DC} does not have this property.
\item \ac{DC} has the ability to produce \ac{quad} surfaces and preserve sharp features, while \ac{MC} produces surfaces consisting of \acp{tri}\footnote{The construction of \acp{tri} is not a drawback of the \ac{MC} method in general, but for our purpose we have to construct \acp{quad} and therefore this point is crucial. If we want to finally produce a \ac{NURBS} surface, we have to rely on \acp{quad}, because \ac{NURBS} have a rectangular topology. \Acp{tri} just do not fit the topology of \ac{NURBS}. Further explanation can be found in \autoref{sec:NURBS} and \autoref{sec:surfaceImpl}} and sharp features are often lost.
\end{itemize}
To come up with these drawback, hybrid methods have been developed: The \acf{DualMC} method and the \acf{DualMT} method. Both methods use ideas from both the \ac{MC} and the \ac{DC} world:
\\
\\
\begin{tabular}{l|l}
\textbf{\ac{MC}} 	& traverse voxels and use a look up table for the creation of faces
					 \\ \hline
\textbf{\ac{DC}} & yoda
\end{tabular}



\todo[inline]{needs to be formulated: No subsubsections or complicated theory, shortly explain main points: hybrid of \ac{DC} \& \ac{MC}, produces manifold surfaces of \acp{quad}, further steps?}
\todo[inline]{clearly state that this part is nothing but an outlook right now!}