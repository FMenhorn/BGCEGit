\subsection{Parametrization of datapoints}
\todo[inline]{just first draft of formulation}
For the least squares fit we need additional information on the datapoints we obtained from surface extraction (see \todo{refer to subsection surface extraction}), which is:
\begin{itemize}
\item To which NURBS--Patch \todo{nomenclature consistent with Erik?} does each datapoint belong to?
\item Which parameters $u,v$ does the datapoint have on this patch?
\end{itemize}
For answering these two questions we used the following approach...
\todo[inline]{shortly explain twoscale DC, projection with QR decomposition, getting parameters from projection}
\subsubsection{Getting a coarse quad mesh \todo{better caption?}}
\todo[inline]{twoscale DC, this is not automatic, nor adaptive, nor topology safe!}
\subsubsection{Projection of datapoints onto quads}
\todo[inline]{making quads plane using least squares (Reference?), otherwise each projection is a non--linear optimization! show picture with both}
\todo[inline]{coarse criterion: Distance to centroid (better: spacetree!)}
\todo[inline]{fine criterion: Projection with QR decomposition}
\subsubsection{Parameter estimation}
\todo[inline]{explain different strategies, comparison of final results would be great (could also be part of third milestone?)}
\todo[inline]{show possible problems}