\clearemptydoublepage

\phantomsection
\addcontentsline{toc}{chapter}{Outline and Overview of the document}

\begin{center}
	\huge{Outline and Overview}
\end{center}




%--------------------------------------------------------------------
\todointern[inline]{is this consistent with the table of contents?}
The purpose of this document is to describe the implementation details of the \emph{CAD-integrated Topology Optimization} software tool along with the theoretical background it relies on. The document is arranged in chapters, covering introduction to the field and the project, background theory and parts of implementation. The chapters are described in more detail below.
\\
\\
%--------------------------------------------------------------------
\noindent {\scshape Chapter 1: Introduction}  \vspace{1mm}

\noindent  This chapter presents an overview of the motivation behind \emph{CAD-integrated Topology Optimization}, including the current state of the field. It also provides general organizational information about project execution, timeline and structure.
\\


%--------------------------------------------------------------------
\noindent {\scshape Chapter 2: Background theory}  \vspace{1mm}

\noindent This chapter introduces the theoretical background for the implementation of the \emph{CAD-integrated Topology Optimization} tool. It consists of five parts, each describing essential background of the topology optimization pipeline. Furthermore, detailed description of selected algorithms used in each step is given.
\\

\noindent {\scshape Chapter 3: Implementation}  \vspace{1mm}

\noindent This chapter provides details on the implementation and structure of the \emph{CAD-integrated Topology Optimization} tool itself. The different parts of the topology optimization and surface-fitting pipeline are presented along with underlying implementation details.
\begin{comment}
\\
\noindent {\scshape Chapter 4: Summary and future work}  \vspace{1mm}

\noindent uhizgufuzutrtszst \todourgent{write something here. Saumi?}
\\
\end{comment}
