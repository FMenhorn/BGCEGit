\acsetup{first-long-format=\itshape}

% class `abbrev': abbreviations:
\DeclareAcronym{DC}{
  short = DC ,
  long  = Dual Contouring ,
  class = abbrev
}
\DeclareAcronym{QEF}{
  short = QEF,
  long  = quadratic error function,
  class = abbrev
}
\DeclareAcronym{MC}{
  short = MC ,
  long  = Marching Cubes ,
  class = abbrev
}
\DeclareAcronym{DualMC}{
  short = DualMC ,
  long  = Dual Marching Cubes ,
  class = abbrev
}
\DeclareAcronym{DualMT}{
  short = DualMC ,
  long  = Dual Marching Tetrahedra ,
  class = abbrev
}
\DeclareAcronym{NURBS}{
  short = NURBS ,
  long  = Non-uniform Rational B-spline ,
  class = abbrev
}
\DeclareAcronym{Bspline}{
  short = B-spline ,
  long  = basis spline ,
  class = abbrev
}
\DeclareAcronym{CADTopOpt}{
  short = CADTopOpt,
  long = CAD-integrated Topology Optimization,
  class = abbrev
}
\DeclareAcronym{CAD}{
  short = CAD,
  long = Computer Aided Design,
  class = abbrev
}
\DeclareAcronym{CSG}{
  short = CSG,
  long = constructive solid geometry,
  class = abbrev
}
\DeclareAcronym{BREP}{
  short = BREP,
  long = boundary representation,
  class = abbrev
}
\DeclareAcronym{STL}{
  short = STL,
  long = stereo lithography,
  class = abbrev
}
\DeclareAcronym{STEP}{
  short = STEP,
  long = standard for the exchange of product model data,
  class = abbrev
}
\DeclareAcronym{IGES}{
  short = IGES,
  long = Initial Graphics Exchange Specification,
  class = abbrev
}
% class `nomencl': nomenclature
\DeclareAcronym{quad}{
  short = quad ,
  long  = quadrilateral face ,
  class = nomencl
}
\DeclareAcronym{tri}{
  short = triangle,
  long = triangular face,
  class = nomencl
}
\DeclareAcronym{patch}{
  short = patch,
  short-plural = es, 
  long  = a patch of rectangular topology,
  long-plural-form = patches of rectangular topology,
  class = nomencl
}
\DeclareAcronym{voxel}{
  short = voxel,
  long = a cube with uniform edge length and a datavalue on each corner of the cube,
  class = nomencl
}
\DeclareAcronym{FirstOrderHermiteData}{
  short = first order Hermite data,
  long = first derivative (gradient respectively) and value of a function at a certain point,
  class = nomencl
}
\DeclareAcronym{SignChangingEdge}{
  short = sign changing edge,
  long = an edge which has vertex values both above and below the isovalue,
  class = nomencl
}