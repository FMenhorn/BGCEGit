\chapter{Introduction}
\label{chapter:Introduction}

In this chapter a motivation for the project and a short description of the problem task is stated; a brief introduction to topology optimisation and the project structure is described.
\section{Motivation}
% Struggle designer engineer 
A common problem in product design is to create a functional structure using as few material as possible. Three decades ago engineering design versions were drawn, prototypes created and experimental test performed. Nowadays, the field of topology optimisation simplifies this process and has become a great help in all fields of engineering. 

Topology optimisation tackles the problem of material distribution in a structure in order to fulfill certain target loads. Several topology optimisation open-source tools exist that are ready to use; a challenge, however, is to incorporate these tools handily in the design process. Triggering these tools with plain CAD files and transferring the solution back to the CAD world is the idea behind this project; in order to keep output CAD files in reasonable sizes smoothing and feature recognition steps are employed.

\section{Important concepts}

\subsection{Computer Aided Design - CAD}

\subsection{Topology Optimisation}

\section{Project structure}
\subsection{Aims and Goals}
The aim of the project is to provide a tool, which allows to perform \textit{Topology Optimisation} without leaving CAD--framework. This includes the reaching of the following goals:
\begin{itemize}
\item Implementation of \textit{Topology Optimisation} using available open source libraries
\item Development of the flexible tool for the conversion of the optimized surface to back to the CAD format.
\end{itemize}
\subsection{Timeline and Structure}
The project is to be completed within 10 months. This period is divided into 4 phases:\\

\textbf{Phase 1:} Getting familiar with the topic and agreement on the project specification.\\

\textbf{Phase 2:} Implementation of the first part of the pipeline (Topology Optimisation from CAD surface using existing tools), investigating the tools and algorithms available for the conversion of the geometry generated after topology optimisation back to CAD format (later referred as \textit{NURBS fitting pipeline}), prototyping (using MATLAB) and evaluating of found solutions.\\

\textbf{Phase 3:} Reimplementing the prototypes, developed on a previous stage, using an open source language, such as Python or C++, extension of the NURBS fitting algorithm to more complex cases, finalising the first part of the pipeline.\\

\textbf{Phase 4:} Implementation of the extended NURBS fitting algorithm, integrating it with the \textit{Topology Optimisation} part and forming of final deliverables.

