Due to the good range of topology optimization software available, we decided to adapt an open-source topology optimizer to our needs, \emph{ToPy}.

\subsection{ToPy library}\label{sec:ToPy}
ToPy \cite{ToPy} is a python library/program, written by William Hunter and documented in \cite{Hunter2009}, implementing the SIMP model and method described in \autoref{subsec:TopOpTheory}. It is based on the 99-line Matlab code by Sigmund's for minimum compliance \cite{sigmund200199}. The program can optimize the previously named problem types: minimum compliance, heat conduction and mechanism synthesis-- in 2D as well as 3D. It uses available open source python libraries, as for example Pysparse and Numpy, leading to improved speed, porta- and scalability. The whole program is steered by an input file which-- with the help of the documentation-- is straightforward to use and easy to adapt. %citation on pysparse and numpy

\subsection{Implementation}
In terms of our implementation, we use ToPy as a blackbox topology optimizer. This means, we launch the program with an input file based on our scenario, let ToPy run and proceed by working with the output of ToPy. The intention is to touch the solver itself as less as possible to be able to just plug in different solvers later on. Implementation-wise that means, that we wrote a program which taking as input a voxelized CAD design in for example stl-format (see \autoref{subsub:STL}), outputting a tpd-file to be used by ToPy. Results of the process can be seen in figure \ref{fig: topyStar}. Here, a star was given as input from a stl-file. We set the voxels in the star's points as fixtures, while we set a load in the middle, in the direction normal to the plane of the star. As can be seen, the optimization process "cuts" away unnecessary material in-between the corners and even in the middle of the material, returning an optimally stiff structure for the chosen remaining volume fraction. 
\begin{figure}
\centering
\begin{subfigure}{
  \includegraphics[width=.2\linewidth]{Pictures/TopOp/Star_Optimized0_Trans.png}}
\end{subfigure}%
\begin{subfigure}{
  \includegraphics[width=.2\linewidth]{Pictures/TopOp/Star_Optimized2_Trans.png}}
\end{subfigure}
\begin{subfigure}{
  \includegraphics[width=.2\linewidth]{Pictures/TopOp/Star_Optimized4_Trans.png}}
\end{subfigure}
\begin{subfigure}{
  \includegraphics[width=.2\linewidth]{Pictures/TopOp/Star_Optimized5_Trans.png}}
\end{subfigure}
\caption{Topology Optimization by ToPy \cite{ToPy}, with minimum compliance. The structure was given by an stl-file and process into input readable by ToPy. Here, fixtures were applied in the corners of the star, and a load in the direction normal to the star was set in the middle.} %which volume fraction?
\label{fig: topyStar}
\end{figure}